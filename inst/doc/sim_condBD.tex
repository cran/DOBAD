\documentclass[12pt]{article}
%\documentclass[article,12pt]{amsart}


%%%%%%%%%%%%%%%%%%%%%%%%%%%%%%%%%%%%%%%%%%%%
% The % is the comment character.
% If you use emacs as a text editor, then the command
% M-x global-font-lock-mode 
% will turn on command highlighting
%%%%%%%%%%%%%%%%%%%%%%%%%%%%%%%%%%%%%%%%%%%%

%%%%%%%%%%%%%%%%%%%%%%%%%%%%%%%%%%%%%%%%%%%
% Here are packages for fonts, symbols, and graphics

%\usepackage{mathrsfs}
\usepackage{amssymb}
\usepackage[english]{babel}
\usepackage{graphicx}
\usepackage{versions}
\usepackage{amsmath}
\usepackage{appendix}

\usepackage{color} %
\usepackage{times} %
\usepackage{amsthm} %
\usepackage{amsfonts} %
\usepackage[small]{caption} %
\usepackage{natbib} %
\usepackage[letterpaper]{geometry} %
%\usepackage{hyperref} %


\usepackage{colortbl}
\definecolor{myGrey}{rgb}{.7,.75,.75}

\bibpunct{(}{)}{;}{a}{}{,} %
\setlength{\leftmargini}{4.8mm} %
\setlength{\leftmargini}{4.8mm} %
\setlength{\leftmarginii}{4.8mm} %
\geometry{hmargin={1.34in,1.14in}, vmargin={1.02in,.99in}}



%%%%%%%%%%%%%%%%%%%%%%%%%%%%%%%%%%%%%%%%%%%
% Here are useful environments. Use them by typing, e.g.
%  \begin{thm} Statement of theorem \end{thm}
% Often followed at some point by \begin{proof} Proof \end{proof}

% \newtheorem{thm}{Theorem}[section]
% \newtheorem{lem}[thm]{\textbf Lemma}
% \newtheorem{cor}[thm]{Corollary}
% \newtheorem{prop}[thm]{\textbf Proposition}
% \newtheorem{crit}[thm]{Criterium}
% \newtheorem{alg}[thm]{Algorithm}


%%%%%%%%%%%%%%%%%%%%%%%%%%%%%%%%%%%%%%%%%
% Here is a different environment. Use it the same way and see what it looks
% like

%\theoremstyle{definition}

% \newtheorem{defn}[thm]{Definition}
% \newtheorem{conj}[thm]{Conjecture}
% \newtheorem{exmp}[thm]{\textbf{Examples}}
% \newtheorem{exe}[thm]{\textbf{Example}}
% \newtheorem{prob}[thm]{Problem}

%%%%%%%%%%%%%%%%%%%%%%%%%%%%%%%%%%%%%%%%%
% Here is a different environment. Use it the same way and see what it looks
% like

% \theoremstyle{remark}

% \newtheorem{rem}[thm]{\textbf{Remark}}
% \newtheorem{note}[thm]{Note}
% \newtheorem{claim}[thm]{Claim}  \renewcommand{\theclaim}{}
% \newtheorem{summ}{Summary}      \renewcommand{\thesumm}{}
% \newtheorem{case}{Case}
% \newtheorem{ack}{ACKNOWLEDGEMENTS}        \renewcommand{\theack}{}

%%%%%%%%%%%%%%%%%%%%%%%%%%%%%%%%%%%%%%%
% Some macros for frequently used commands

\def\R{\mathbb{R}}
\def\to{\rightarrow}
\def\der#1#2{\frac{\partial #1}{\partial #2}}  %% This is a partial derivative
\def\ip#1#2{\left<#1,#2\right>}  %% This is an inner product
\def\inv#1{\frac{1}{#1}}


%%%%%%%%%%%%%%%%%%%%%%%%%%%%%%%%%%%%%%%
% Formatting stuff

\renewcommand{\baselinestretch}{1.5}
\setlength{\textwidth}{167mm} \addtolength{\hoffset}{-22mm}

%%%%%%%%%%%%%%%%%%%%%%%%%%%%%%%%%%%%%%
% No box at the end of proofs

%\renewcommand{\qedsymbol}{}

%%%%%%%%%%%%%%%%%%%%%%%%%%%%%%%%%%%%%%
% Now for the actual document

%%%
%%%%%%%%%%%%%%macros for this particular document
%%%

\def\Z{\mathbb{Z}}


%I am so confused.  \alpha_1, \sigma_1, \phi_1 are broken but _2 work. what?
\def\la{\lambda}
\def\ka{\kappa}
\def\elrt#1#2{ e^{- \la (#1 -#2)rt} }

%\includeversion{self}
\excludeversion{self}

%%Do not delete the following line. It is used by the R package documentation
% \VignetteIndexEntry{ Simulate BDI process Conditional on Partial Observations }
%%Do not delete the above line. It is used by the R package documentation
\title{DOBAD Package:  simulation of BDI process conditional on discrete observations}

\author{Charles Doss}
\date{September 2009}

\usepackage{Sweave}
\begin{document}



\begin{titlepage}
\maketitle
\end{titlepage}


\vspace{-3.7mm} %






\part{Simulation of a Linear BDI Process, Conditional on Observing it
  at Discrete Times}

We are demonstrating the use of the
\begin{verb}
DOBAD
\end{verb}
package's function for conditionally simulating a birth-death
process,using the methods of \citet{DSHKM2010EM}.  Call the process
$\{X(t)\}_{t \in \R}$; we will simulate it conditional upon seeing
data which is the value of the process at a finite number of discrete
time points.  That is, for times $0=t_0, t_1, \ldots, t_n$, we see the
state of the process, $X(t_i)$.  Thus the data $D$ is $2$ parts: a
vector of the times $t_i$, $i= 0, \ldots, n$ and a vector of states at
each of those times, $s_i$, for $i=0, \ldots, n$ (where $X(t_i) =
s_i$.

\begin{Schunk}
\begin{Sinput}
> library(DOBAD)
\end{Sinput}
\end{Schunk}

Generate a chain, the ``truth'' that we would not observe in practice:
\begin{Schunk}
\begin{Sinput}
> L <- 0.3
> m <- 0.5
> nu <- 0.4
> set.seed(112)
> unobservedChain <- birth.death.simulant(t = 5, X0 = 11, lambda = 0.3, 
+    mu = 0.5, nu = 0.4)
> unobservedChain
\end{Sinput}
\begin{Soutput}
An object of class "BDMC"
Slot "states":
 [1] 11 12 11 10  9 10 11 10 11 10  9 10  9 10  9 10  9  8  7  8  9 10  9  8  7
[26]  8  9  8  7  6  5  6  7  8  7  8  7  8  7  8  7  6  5  4

Slot "times":
 [1] 0.0000000 0.1304171 0.1378384 0.1874095 0.3413327 0.4836734 0.5151733
 [8] 0.5570117 0.7665553 0.9320789 1.0212908 1.1413929 1.1998740 1.2975232
[15] 1.3406924 1.3665537 1.3994869 1.6049192 1.6333970 1.7129540 1.7668332
[22] 1.8285686 1.8363130 1.9739581 1.9851389 2.0814101 2.0854210 2.3046697
[29] 2.4094295 2.5627537 2.6068091 3.1822596 3.3482861 3.3844540 3.4134403
[36] 3.4503969 3.4519932 3.9011731 4.1374942 4.2672352 4.3789622 4.5547798
[43] 4.6734574 4.7213272

Slot "T":
[1] 5
\end{Soutput}
\end{Schunk}
Then fix some ``observation times'' and ``observe'' the chain:
\begin{Schunk}
\begin{Sinput}
> times <- c(0, 0.21, 0.62, 0.73, 1.44, 1.95, 3.56, 4.17)
> obsData <- getPartialData(times, unobservedChain)
> obsData
\end{Sinput}
\begin{Soutput}
An object of class "CTMC_PO_1"
Slot "states":
[1] 11 10 10 10  9  9  7  7

Slot "times":
[1] 0.00 0.21 0.62 0.73 1.44 1.95 3.56 4.17
\end{Soutput}
\end{Schunk}
Now, we do a conditional simulation:
\begin{Schunk}
\begin{Sinput}
> nsims <- 5
> condSims <- sim.condBD(N = nsims, bd.PO = obsData, L = L, m = m, 
+    nu = nu)
> condSims[1]
\end{Sinput}
\begin{Soutput}
[[1]]
An object of class "BDMC"
Slot "states":
 [1] 11 10  9 10 11 10 11 10  9 10  9 10  9  8  7  8  9  8  7  6  7

Slot "times":
 [1] 0.00000000 0.05224893 0.44767799 0.57874166 0.76491147 0.86669301
 [7] 0.99362847 1.17552882 1.41615881 1.64403070 1.68311572 1.75535090
[13] 1.91845847 2.06208393 2.10971694 2.29007387 2.31764070 2.59837720
[19] 2.70438486 3.88598190 4.08667118

Slot "T":
[1] 4.17
\end{Soutput}
\begin{Sinput}
> condSims[4]
\end{Sinput}
\begin{Soutput}
[[1]]
An object of class "BDMC"
Slot "states":
 [1] 11 10  9 10 11 10  9  8  9 10 11 10 11 10  9  8  7  8  7  6  7  8  7  8  7
[26]  8  7  8  7

Slot "times":
 [1] 0.00000000 0.02210702 0.44783042 0.46187799 0.64255612 0.69685008
 [7] 0.77231358 0.85598923 1.42694811 1.44175639 1.46869934 1.63690122
[13] 1.66096808 1.77043653 1.80831279 2.17136625 2.45274890 2.55077418
[19] 2.59310847 2.60258622 2.71888487 2.74860598 2.96281893 2.98744978
[25] 3.08537681 3.70667131 3.89807045 4.01906935 4.09732896

Slot "T":
[1] 4.17
\end{Soutput}
\end{Schunk}

\bibliographystyle{biom} %
\bibliography{DOBADbiblio}


\end{document}





